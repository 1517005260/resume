% !TeX TS-program = xelatex

\documentclass{resume}
\usepackage{ctex}
\usepackage{hyperref}
\usepackage[utf8]{inputenc}

\ResumeName{高菻铠}

\begin{document}

\ResumeContacts{
    (+86)137-6159-6615,% 
    \ResumeUrl{mailto:linkaigao77@gmail.com}{linkaigao77@gmail.com},%
    \ResumeUrl{https://github.com/1517005260}{github.com/1517005260}%
}

\ResumeTitle

\section{教育经历}
\ResumeItem
[华东理工大学|本科生]
{华东理工大学}
[\textnormal{计算机与金融双学位}]
[2022.09—2026.06]

\textbf{专业必修加权均分:90.83,GPA:3.77(专业前2)},多次获学业奖学金。

\textbf{主修课程:}高等数学上(89),高等数学下(100),线性代数(91),概率论与数理统计(99),离散数学(97),计算机组成原理(99),算法与数据结构(93),数据库原理(94),金融计量学(92)。

\section{技术能力}
\begin{itemize}
    \item \textbf{编程语言}: 熟悉 Python, Java, JavaScript, C, C++,能够灵活适应项目需求进行语言切换。
    \item \textbf{开发工具}: 熟悉 Linux, Shell, Vim, Git, GitHub, GitLab, Docker,具备一定的服务器部署和容器化技术的实践经验。
    \item \textbf{大模型实践}: 掌握大模型的部署、微调及 RAG 工作流。
    \item \textbf{金融知识}: 具备金融基础知识,能够将技术应用于金融领域。
\end{itemize}

\section{项目经历}

\textbf{政务大模型} \hfill \textbf{大创主持人}
\begin{itemize}
    \item 负责项目管理,协调团队工作,推动项目按时完成。
    \item 参与商学院、学工部等政策文件的收集,完成数据分类和预处理。
    \item 搭建 RAG 工作流,使用 Ollama-OneAPI-FastGPT 框架。
    \item 使用 Llama-Factory 微调 7B 和 14B 开源模型,提高模型性能。
    \item 计划结合 GraphRAG 和知识图谱技术,提升召回率和精确度。
\end{itemize}

\textbf{\href{https://app6534.acapp.acwing.com.cn/}{基于 Django 的多人联机小游戏}} \hfill \textbf{全栈开发}
\begin{itemize}
    \item 使用 JavaScript 实现简化版游戏引擎,模拟 Unity 3D 引擎。
    \item 基于 WebSocket 实现多人联机对战的实时双向通信。
    \item 集成 Oauth2 进行第三方登录认证。
    \item 配置 Nginx 反向代理,绑定自定义域名,优化用户体验。
\end{itemize}

\textbf{\href{http://116.198.216.39/}{基于 Spring Boot 的校园论坛}} \hfill \textbf{全栈开发}
\begin{itemize}
    \item 基于 MVC 架构,设计并实现权限管理、通知和搜索等功能模块。
    \item 使用三指针算法与前缀树完成敏感词过滤。
    \item 利用 Redis 实现点赞、关注和日活跃统计功能,提升用户体验。
    \item 使用 Kafka 消息队列异步处理系统通知。
    \item 使用 ElasticSearch 进行分词索引,优化搜索性能。
    \item 通过 Jmeter 和 Caffeine 进行并发测试,提升系统的稳定性和吞吐量。
\end{itemize}

\textbf{大模型情感分析} \hfill \textbf{学术训练项目}
\begin{itemize}
    \item 结合大语言模型进行情感分析,研究情感因素对筹资绩效的影响。
    \item 使用高频数据进行情感分类,将结果作为回归分析中的独立变量。
    \item 探讨情感维度(如同情、紧迫感)对捐赠行为的影响,提升模型的解释力。
    \item 引入情感强度和变化等细粒度标签,进一步研究情感变量的作用。
\end{itemize}

\section{个人总结}

\begin{itemize}
    \item 热爱技术,渴望在未来深造中进一步提升专业知识和实践能力,期待能够探索更深层次的计算机技术前沿领域。
    \item 自主学习能力强,能够迅速适应新技术和新工具,积极主动参与各类项目,保持对技术进步的敏感度。
    \item 性格乐观向上,具备较强的自我驱动力和抗压能力,始终保持对技术的热情,并愿意通过持续学习和实践提升自我。
    \item 英语交流能力较好(CET-6 539),能够阅读英文文献及技术资料,并积极参与技术社区的讨论与学习。
    \item 具备一定的 Linux 使用经验、软件开发经验和大模型实践经验,能够有效结合多学科知识,推动项目落地。
\end{itemize}

\end{document}
