% !TeX TS-program = xelatex

\documentclass{resume}
\usepackage{ctex}
\usepackage{hyperref}
\usepackage[utf8]{inputenc}

\ResumeName{高菻铠}

\begin{document}

\ResumeContacts{
    (+86)137-6159-6615,%
    \ResumeUrl{mailto:linkaigao77@gmail.com}{linkaigao77@gmail.com},%
    \ResumeUrl{https://github.com/1517005260}{github.com/1517005260}%
}

\ResumeTitle

\section{教育经历}
\ResumeItem
[华东理工大学|本科生]
{华东理工大学}
[\textnormal{计算机与金融双学位}]
[2022.09—2026.06]

\textbf{专业必修加权均分:90.83,GPA:3.77(专业前2)},多次获学业奖学金。

\textbf{主修课程:}高等数学上(89),高等数学下(100),线性代数(91),概率论与数理统计(99),离散数学(97),计算机组成原理(99),金融计量学(92)。

\section{技术能力}
\begin{itemize}
    \item \textbf{编程语言}: Python, Java, JavaScript, C, C++,能够灵活切换语言满足项目需求。
    \item \textbf{开发工具}: Linux, Shell, Vim, Git, GitHub, GitLab, Docker,具备服务器部署和容器化技术实践经验。
    \item \textbf{大模型实践}: 熟练掌握大模型部署、微调及 RAG 工作流。
    \item \textbf{金融知识}: 拥有金融基础知识,擅长将技术与金融领域应用相结合。
\end{itemize}

\section{项目经历}

\textbf{政务大模型} \hfill \textbf{大创主持人}
\begin{itemize}
    \item 主导项目管理,协调团队工作,确保项目按时交付。
    \item 分模块收集商学院、学工部等相关政策文件,完成数据分类与预处理。
    \item 利用 Ollama-OneAPI-FastGPT 框架,搭建 RAG 工作流。
    \item 使用 Llama-Factory 微调 7B 和 14B 开源模型,针对性提升模型性能。
    \item 计划使用 GraphRAG 结合知识图谱技术,进一步提高召回率与精确性。
\end{itemize}

\textbf{\href{https://app6534.acapp.acwing.com.cn/}{基于 Django 的多人联机小游戏}} \hfill \textbf{全栈开发}
\begin{itemize}
    \item 使用 JavaScript 模拟 Unity 3D 引擎,开发简化版游戏引擎。
    \item 基于 WebSocket 实现实时双向通信,支持多人联机对战。
    \item 集成 Oauth2 实现第三方登录认证。
    \item 配置 Nginx 反向代理,绑定自定义域名,优化用户体验。
\end{itemize}

\textbf{\href{http://116.198.216.39/}{基于 Spring Boot 的校园论坛}} \hfill \textbf{全栈开发}
\begin{itemize}
    \item 基于 MVC 架构,设计并实现权限管理、核心功能、通知和搜索模块。
    \item 使用三指针算法与前缀树实现社区内容的敏感词过滤。
    \item 利用 Redis 实现点赞、关注、日活跃统计等功能,提升用户交互体验。
    \item 使用 Kafka 消息队列异步处理系统通知。
    \item 使用 ElasticSearch 进行分词索引,优化社区内容搜索效率。
    \item 通过 Jmeter 和 Caffeine 进行并发测试,确保高负载下的系统稳定性,使网站吞吐量翻倍。
\end{itemize}

\textbf{大模型情感分析} \hfill \textbf{学术训练项目}
\begin{itemize}
    \item 结合大语言模型进行情感分析,整合社会行为特征,分析情感因素对筹资绩效的影响。
    \item 使用高频数据集进行情感分类,将结果作为回归分析的独立变量。
    \item 探讨情感维度(如同情、紧迫感)对捐赠行为的影响,验证其对模型解释力的提升。
    \item 引入情感强度和变化等细粒度标签,进一步细化情感变量的作用。
\end{itemize}

\section{个人总结}

\begin{itemize}
    \item 乐观开朗,自驱力强,具备良好沟通能力与团队合作精神。
    \item 具备英语工作交流能力(CET-6 539),有阅读英文开源文档习惯。
    \item 拥有 Linux 使用经验、软件开发经验、大模型实践经验,善于技术写作,关注前沿技术。
\end{itemize}

\end{document}
